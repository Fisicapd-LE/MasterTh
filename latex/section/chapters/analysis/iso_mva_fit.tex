\subsection{Isolation MVA fit}
\label{subsec:iso_mva_fit}

We now present the results of the fit of the isolation based MVA. 
The plots shown in \autoref{gr:isomva_barrel} and \autoref{gr:isomva_endcap} are valid only for non fully isolated \Pgm, as these are the only ones where the input variables are defined.

As an input of the fit we have the weight of the fake category computed before, as a constraint to help the fit converge, especially in the non direct components that are too correlated to give meaningful results alone.

After the fit we use the isolation ratio (first column) from \autoref{tab:iso_counts} to estimate the amount of isolated \Pgm in each category and we add these to the count.

After this the numbers are treated like those from the previous MVA, as an hint and constraint for the final impact parameter fit.

The results of the fit are shown in \autoref{gr:isomva_barrel} and \autoref{gr:isomva_endcap}.
As we can see, the model does not have a perfect agreement with the data.

\inputgraph{isomva_barrel}
\inputgraph{isomva_endcap}
