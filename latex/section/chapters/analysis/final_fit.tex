\subsection{Final fit}
\label{subsec:final_fit}

The main analysis of the thesis is the impact parameter fit. 

Unlike the MVA, here our models are analytical, with parameters that we estimated from MC. 
To avoid sistematic errors due to an imprecise estimation of the parameter of the models, we allow the model parameters to float around their computed values.
We add instead gaussian constraints in each parameter, to force it to stay close to its initial value, with the constraint sigma taken from the estimated error in the models fit.
This removes a source sistematic errors at the cost of increasing the statistical one and the complexity of the fit.

Since we are forced to assume the same model for signal and pileup, we cannot allow the model parameters to float differently in the two subsamples. 
The two models used in the fit have thus shared parameters and the fit is executed simultaneously for both of them.
This is the reason why the second method in \autoref{subsec:fit_method} was discarder: if we lowered the weight of the pileup \Pgm, the shared parameters would converge to values accomodating the signal much more, leaving the pileup inappropriate parameters that do not follow its shape.

\inputgraph{imppar_barrel}
\inputgraph{imppar_endcap}

The fits of each window of \pt, \psrap and $\mathrm{d}z$ are shown in \autoref{gr:imppar_barrel} and \autoref{gr:imppar_endcap}.
In the figures the read area represents fake \Pgm, the green area \Pgm from \Pqc quarks and the white area the \Pgm from \Pqb quarks.

Here result, still in flux.

\draft{%
The number of \Pqb \Pgm is then summated in all the windows.%
And in the end the ratio is computed.%
}

\draft{%
The total number of \Pgm for signal or background is calculated as a sum of the windows, and its error as a quadrature sum of the errors. %
This should work because they are fully independent samples, but I would prefer a good talk on errors here, and also with the ratio error.%
}
