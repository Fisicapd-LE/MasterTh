\subsection{Fake \Pgm impact parameter model}
\label{subsec:fake_model}

We showed before that the fake \Pgm impact parameter model is different in the each of the MC samples. 
This also means that we cannot rely on it when fitting on data, as it is very likely to be different as well.

Our MVA has on the other hand given a good way to discriminate fake \Pgm from real ones: if we take \Pgm whose fake MVA value is low enough (less than a threshold, selected to be 0.2 from the MVA output distributions), we are selecting a subsample that is largely composed of fake \Pgm.
When using this method a problem may arise: we need to be certain that the mva cut will not alter the impact parameter distribution.

Using Monte Carlo samples, we compute the ratio the fake impact parameter distribution and the distribution of \Pgm failing the MVA selection.
We want this ratio to be costant, at least in the \dxy region that we consider.

The two distributions superimposed are shown in \autoref{gr:mva_fakes}, while their ratio in \autoref{gr:mva_fakes_ratio}.

\inputgraph{mva_fakes}
\inputgraph{mva_fakes_ratio}

The ratio was found to be sufficiently constant (high probability for the costant fit), so the method was successfully used in the following parts of the analysis.
