\section{Introduction}

A precise measurement of the $\xsec{\Pp\Pp\to\Pqb\Paqb X}$ is of extreme importance for testing next to leading order Quantum Chromodynamics calculations.
Previous analysis at lower energies have been made by the UA1 collaboration at the cern $S\overline{p}ps$ collider, with a $\sqrt{s} = \SI{0.63}{\TeV}$ (\cite{bib:spps1} and \cite{bib:spps2}) and by the the CDF and D0 experiments at Tevatron, at $\sqrt{s}=\SI{1.8}{\TeV}$ and $\sqrt{s}=\SI{1.96}{\TeV}$ (\cite{bib:CDF1} to \cite{bib:D0last}).
More recently there have been measurements in LHC at $\sqrt{s}=\SI{7}{\TeV}$ from the LHCb experiment, using semiinclusive decays (\cite{bib:LHCb_bb}), and from CMS, using fully reconstructed $\PBp\to\PJgy\PK\to\Pgmp\Pgmm\PK$ decays (\cite{bib:CMS_bb}).

%\draft{why?}

In our thesis we present a novel method for the measurement of this cross-section at \SI{13}{\TeV} using semileptonic decays, that does not depend on measure of the luminosity, the branching ratio of the processes or the muon reconstruction efficiencies.
The measurement uses the ratio the numbers of semileptonic decays on pileup and on a vertex with an already fully reconstructed B-meson.
All the usual normalizations will be simplified by virtue of using the same process on the same events.
We expect instead our uncertainty to be dominated by the inaccuracies of the models from the Monte Carlo and the similarities between their parameters.
We will develop new tags based on Machine Learning techniques to help the fit converge and minimize fit uncertainties.
In the end the result presented will be ratio between the integral cross-section of the $\Pp\Pp\to\Pqb\Paqb X$ and the "minimum-bias cross-section" in the CMS experiment, an already well measured value.
We will also discuss on the feasability of this new method and whether it is possible to achieve better accuracies than the previously used ones.

\draft{Ho trovato articoli di CMS (https://arxiv.org/pdf/1101.0131.pdf) e LHCb (https://www.sciencedirect.com/science/article/pii/S0370269310012074?via\%3Dihub) che facevano la stessa misura mia usando, dando come motivazione "confronto con la teoria in nuove finestre di energia". E' questo l'unico motivo o ce ne sono altri? Il vantaggio del mio metodo che vedo dagli articoli e' che io non ho il grande errore che hanno loro sulla luminosita' (anche se alla fine a me rientra nella misura di pp->X se voglio la vera e propria sezione d'urto).}

\draft{Qui discuterei delle misure precedenti, direi che c'e' anche la stima NLO, spiegherei vantaggi e svantaggi del mio metodo rispetto ai metodi tradizionali di cercare un decadimento e usare BR+lumi, probabili fonti di errore per me (distribuzioni MC + correlazioni risultati probabilmente)}

\draft{pp->X e' misurato bene?}
