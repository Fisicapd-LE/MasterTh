\subsection{Reconstructed vertices}
\label{subsec:reco_ver}

We said in the prevoius section that we need a fully reconstructed B-meson decay vertex for our analysis.
This is because we need to be certain that the mother particle is actually a B-meson and also because we need to be able to exclude the tracks coming from this decay from the count of \Pgm for the analysis.

The chosen decays to be reconstructed are:
\begin{itemize}
	\item $\PBp\to\PJgy\PKp\to\Pgmp\Pgmm\PKp$
	\item $\PBz\to\PJgy\PKst\to\Pgmp\Pgmm\PKpm\Pgpmp$
	\item $\PsB\to\PJgy\Pgf\to\Pgmp\Pgmm\PKp\PKm$
\end{itemize}

This decay were chosen because they leave a very distinctive track in the CMS tracker (2 \Pgm with invariant mass close to that of the $\PJgy$, one extra track for the $\PBp$ or two extra tracks for the $\PBz$ and $\PsB$, all forming a common vertex).
They also the decays that are often used in CP violation measures, allowing us to use large Monte Carlo samples that are produced for those analysis.
\draft{Other?}

The \Pgm from the \PJgy are of course removed from the count of the \Pgm for the analysis.
