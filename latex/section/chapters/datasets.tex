\section{Samples used}
\label{sec:data}

During the analysis we used 4 different datasets, 3 Monte Carlo samples as a model for the fit and one data sample, all from the year 2016.

The Monte Carlo samples are dataset generated for other parallel analysis.
Each of them is generated so that every event contains at least one secondary vertex of a specific decay, which is used as a label for the dataset.
The used datasets are
\begin{itemize}
	\item $\PsB\to\PJgy\Pgf\to\Pgmp\Pgmm\PKp\PKm$, \num{\sim 100e6} events, \num{2.825e7} after secondary vertex reconstruction
	\item $\PBp\to\PJgy\PKp\to\Pgmp\Pgmm\PKp$, \num{\sim 20e6} events, \num{7.986e6} after SV reconstruction
	\item $\PBz\to\PJgy\PKst\to\Pgmp\Pgmm\PKpm\Pgpmp$, \num{\sim 73e6} events, \num{1.9734e7} after SV reconstruction
\end{itemize}

The data sample is the union of all the so called "Charmonium" datasets collected in the year 2016.
This means that they contain only events that passed a specific set of High Level Triggers (HLT).
Every high level trigger is a set of selections that can be computed extremely fast in a computer farm, as these triggers are executed online during the acquisition and are used to decide if a specific event is worth being stored or not.
In the case of the Charmonium dataset the triggers are designed to ensure the presence of at least one charmonium meson decayin ito two \Pgm inside the event.

In each dataset we had to chose one of the three possible decays (as explained in the previous section) to reconstruct.
In the MC samples we simply reconstructed the required decay, to maximize the number of events not rejected.
In the data sample we instead chose to reconstruct the $\PBp\to\PJgy\PKp$ decay, as of the three decays it's, the most likely to occur.\\

After a first skim, that only selected events with the required decay, some additional selections are applied on the events.

The first one is a selection on the HLT, to reduce background. 
We require that each event passed a trigger that is internally called HLT\_DoubleMu4\_JpsiTrk\_Displaced.
The trigger requires that in the event there are:
\begin{itemize}
	\item Two \Pgm with total invariant mass close to that of the \PJgy (in a window of \SI{150}{\MeV} around it), forming a common vertex fitted with a probability $P>\SI{10}{\percent}$
	\item The vertex has a distance from the beam spot $L_{xy}$ in the transversal plane greater than 3 times its error.
	\item The cosine of the angle between $L_{xy}$ and the total momentum of the \Pgm is greater than 0.9 (we don't want this cut to be strict, since it's considering only the \PJgy momentum instead of the full B-meson)
	\item An extra track, with $\pt > 0.8$, compatible with the \Pgm's vertex with a $\frac{\chi^2}{\text{NDF}} > 10$. No extra requirements are made on this vertex
\end{itemize}
\draft{Apetto ancora alessio per conferma sui numeri}

The second set of cuts applied are quality cuts for the main secondary vertex at the offline analysis stage.
These cuts include
\begin{itemize}
	\item $\cos{\alpha}< 0.99$, with $\alpha$ angle between the secondary vertex direction (computed as the 3d distance between \Pp\Pp interaction point and B decay vertex) and the total momentum of the tracks from the decay. This is a measure of how much momentum we are missing from the decay, to notice potential missing tracks.
	\item Vertex fit probability $ P > \SI{10}{\percent}$.
	\item Distance from the beam axis greater than 3 times the error on it. Since B-meson have a relatively long mean life, with this cut we can reject combinatorial background from prompt \PJgy production.
	\item Total $p_T$ of the meson greater than \SI{10}{\GeV}.
	\item Total $p_T$ of the \PJgy from the decay greater than \SI{8}{\GeV}.
	\item Each \Pgm from the \PJgy decay with $p_T$ greater than \SI{4}{\GeV} and $\eta$ smaller than \num{2.2}.
\end{itemize}

Table \ref{tab:ssb_cuts} shows a summary of the efficiency of the cuts applied on the secondary vertex.

\inputtab{ssb_cuts}

\inputgraph{bmass}

Further cuts are applied on the additional \Pgm collected in the events.
We require the \Pgm to have $p_T>\SI{4}{\GeV}$ and error on the impact parameter $E_{xy} < \SI{0.005}{\cm}$.
We can see in Table \ref{tab:muon_cuts} the effect these selection make on signal or pileup \Pgm.
Additionally, the last rows show the abundance of signal and pileup \Pgm, after selections, in the $p_T$ and $\eta$ windows we use during the analysis.

\inputtab{muon_cuts}

\subload{checks}
